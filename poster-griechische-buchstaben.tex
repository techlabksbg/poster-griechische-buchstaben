\documentclass[a1,portrait]{a0poster}


\usepackage[svgnames]{xcolor} % Specify colors by their 'svgnames', for a full list of all colors available see here: http://www.latextemplates.com/svgnames-colors
\usepackage{graphicx}
\usepackage{adjustbox}

\setlength{\voffset}{2cm}


\begin{document}
%\begin{minipage}[b]{1\linewidth}
    \begin{center}
        \VERYHuge \sf \textbf{Das griechische Alphabet} \color{Black}\\ % Title
%        \VeryHuge{in der Mathematik}\\[2.4cm] % Subtitle
    \end{center}
 %   \end{minipage}

\vspace{1cm}
\Huge
\def\arraystretch{1.05}
\begin{tabular}{lllp{0.74\linewidth}}
%    Gross & Klein & Aussprache  & Gebrauch \\
    A$\,\,$  & $\alpha$ $\,\,\,$ & Alpha & \Large Winkel, $\alpha$-Strahlung (Heliumkerne), Längenausdehnungskoeffizient $\alpha$ \\
    B & $\beta$ & Beta & \Large Winkel, $\beta^-$-Strahlung (Elektronen), Flächenausdehnungskoeffizient $\beta$\\
    $\Gamma$ & $\gamma$ & Gamma & \Large Winkel, $\gamma$-Strahlung (Photonen), Gravitationskonstante $\Gamma$\\ %, Volumenausdehungskoeffizient $\gamma$\\
    $\Delta$ & $\delta$ & Delta & \Large Dreieck, z.B. $\Delta ABC$, Differenz, z.B. $m=\frac{\Delta y}{\Delta x}$\\  
    E & $\varepsilon$ & Epsilon & \Large $\varepsilon$ steht oft für eine sehr kleine positive reelle Zahl\\
    Z & $\zeta$ & Zeta & \Large Riemannsche $\zeta$-Funktion: $\displaystyle \zeta(s) = \sum_{n=1}^{\infty} \frac{1}{n^s}$ mit  $\textrm{Re}(s)>1$\\[-6mm]
    H & $\eta$ & Eta & \Large Wirkungsgrad $\eta$ (bei Wärmekraftmaschinen)\\
    $\Theta$ & $\vartheta$ & Theta & \Large Temperatur in Grad Celsius\\
    I & $\iota$ & Iota & \Large \\
    K & $\kappa$ & Kappa & \Large Krümmungsmass $\kappa$, Adiabatenexponent $\kappa = \frac{C_p}{C_v}$ (Gasgesetze)\\
    $\Lambda$ & $\lambda$ & Lambda & \Large Streckungsfaktor $\lambda$ (Geometrie), Wellenlänge $\lambda$, Zerfallskonstante $\lambda$\\
    M & $\mu$ & Mü & \Large Mittelwert $\mu$, Massprefix (micro) für Millionstel: 1 $\mu$m = $10^{-6}$ m\\
    N & $\nu$ & Nü & \Large Frequenz $\nu$ bei Wellen, Symbol für das Neutrino $\nu$ \\
    $\Xi$ & $\xi$ & Xi & \Large \\
    O & $o$ & Omikron $\,\,$ & \Large \\
    $\Pi$ & $\pi$ & Pi & \Large Kreiszahl $\pi = 3.14159265358979\dots$, Produktzeichen $\prod$\\
    P & $\rho$ & Rho & \Large Symbol für Dichte $\rho$ in der Physik.\\
    $\Sigma$ & $\varsigma$ & Sigma & \Large Summenzeichen $\sum$, Standardabweichung $\sigma$ (Statistik)\\
    T & $\tau$ & Tau & \Large Periodendauer $\tau$ einer Schwingung\\
    Y & $\upsilon$ & Ypsilon & \Large \\
    $\Phi$ & $\varphi$ & Phi & \Large Winkel, Euler-$\varphi$-Funktion, Goldener Schnitt $\Phi=\frac{\sqrt{5}+1}{2} = 1.618033988\ldots$\\
    X & $\chi$ & Chi & \Large $\chi$ Symbol für die Chi-Quadrat Verteilung (Statistik)\\
    $\Psi$ & $\psi$ & Psi & \Large Wellenfunktion $\psi$ in der Quantenphysik\\
    $\Omega$ & $\omega$ & Omega & \Large $\Omega$ Ohm (Widerstand), Winkelgeschwindigkeit $\omega$, kleinste Ordinalzahl $\omega$\\
\end{tabular}

\vspace{1.6cm}
\begin{center}
\adjustbox{valign=M}{\includegraphics[width=4cm]{by-nc-sa.pdf}} 
\adjustbox{valign=M}{\small \sf Ivo Blöchliger} $\quad$
%\adjustbox{valign=M}{{\small \tt https://github.com/techlabksbg/poster-griechische-buchstaben}}
\adjustbox{valign=M}{{\small \sf \LaTeX{} sources on GitHub:}}
\adjustbox{valign=M}{\includegraphics[height=2cm]{qrcode-github.png}} 
\end{center}

\end{document}

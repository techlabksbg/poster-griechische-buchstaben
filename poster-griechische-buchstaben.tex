\documentclass[a1,portrait]{a0poster}


\usepackage[svgnames]{xcolor} % Specify colors by their 'svgnames', for a full list of all colors available see here: http://www.latextemplates.com/svgnames-colors


\begin{document}
%\begin{minipage}[b]{1\linewidth}
    \begin{center}
        \VERYHuge \color{NavyBlue} \textbf{Das griechische Alphabet} \color{Black}\\ % Title
        \VeryHuge{in der Mathematik}\\[2.4cm] % Subtitle
    \end{center}
 %   \end{minipage}

\huge
\def\arraystretch{1.2}
\begin{tabular}{lllp{0.6\linewidth}}
%    Gross & Klein & Aussprache  & Gebrauch \\
    $A$ $\quad$ & $\alpha$ $\qquad$ & Alpha & \Large Variable für Winkel \\
    $B$ & $\beta$ & Beta & \Large Variable für Winkel \\
    $\Gamma$ & $\gamma$ & Gamma & \Large Variable für Winkel. Gravitationskonstante $\Gamma$. \\
    $\Delta$ & $\delta$ & Delta & \Large Dreieck, z.B. $\Delta ABC$. Differenz, z.B. $m=\frac{\Delta y}{\Delta x}$\\  
    $E$ & $\epsilon$, $\varepsilon$ & Epsilon & \Large Steht oft für eine sehr kleine positive reelle Zahl.\\
    $Z$ & $\zeta$ & Zeta & \Large \\
    $H$ & $\eta$ & Eta & \Large \\
    $\Theta$ & $\theta$ & Theta & \Large Winkel in Polarkoordinaten\\
    $I$ & $\iota$ & Iota & \Large \\
    $K$ & $\kappa$ & Kappa & \Large \\
    $\Lambda$ & $\lambda$ & Lambda & \Large Streckungsfaktor, Proportionalitätskonstante\\
    $M$ & $\mu$ & Mü & \Large \\
    $N$ & $\nu$ & Nü & \Large \\
    $\Xi$ & $\xi$ & Xi & \Large \\
    $O$ & $o$ & Omikron $\quad$ & \Large \\
    $\Pi$ & $\pi$ & Pi & \Large Verhältnis von Kreisumfang zu Kreisdurchmesser.\\
    $P$ & $\rho$ & Rho & \Large Symbol für Dichte in der Physik.\\
    $\Sigma$ & $\sigma$, $\varsigma$ & Sigma & \Large Summenzeichen $\sum$, Vorzeichenfunktionen $\sigma$.\\
    $T$ & $\tau$ & Tau & \Large \\
    $Y$ & $\upsilon$ & Upsilon & \Large \\
    $\Phi$ & $\phi$, $\varphi$ & Phi & \Large Variable für Winkel.\\
    $X$ & $\chi$ & Chi & \Large \\
    $\Psi$ & $\psi$ & Psi & \Large \\
    $\Omega$ & $\omega$ & Omega & \Large Mass für Widerstand (Ohm), Winkelgeschwindigkeit $\omega$\\
\end{tabular}
\end{document}
